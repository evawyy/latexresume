%arara: xelatex

\documentclass{resume}
\usepackage{ctex}
% \usepackage{zh_CN-Adobefonts_external} % Simplified Chinese Support using external fonts (./fonts/zh_CN-Adobe/)
%\usepackage{zh_CN-Adobefonts_internal} % Simplified Chinese Support using system fonts

\begin{document}
\pagenumbering{gobble} % suppress displaying page number

\name{王胤雅}

\basicInfo{
	\email{yinyawang25@m.fudan.edu.cn} \textperiodcentered\
	\phone{(+86) 15907464032} \textperiodcentered\
	% \linkedin[billryan8]{https://www.linkedin.com/in/billryan8}
}

\section{\faFemale\ 个人总结}
本人在校期间学习成绩优异,积极乐观,对数学充满热情,且具有广泛的数学分支知识。注重通过计算机技术提升学习和生活效率,并且希望未来能够继续深耕于热爱的学术领域,探索更深层次的数学理论与应用。

\section{\faGraduationCap\ 教育经历}
\datedsubsection{\textbf{北京师范大学},北京,中国}{2019 -- 2025}
本科,数学与应用数学,2025年6月毕业,期间获得5次学业奖学金。
\datedsubsection{\textbf{复旦大学},上海,中国}{2025 -- 至今}
直博,数学,上海数学与交叉学科研究院,预计2030年6月毕业。

\section{\faBook\ 学业成绩}
本科:
\begin{itemize}
	\item GPA 3.6/4.0,排名前20\%(29/140);
	\item 修读20+门纯数学课程,包括泛函分析(95分),马氏过程选讲(93分),实变函数(93分)等;
	\item 修读2门应用数学课程,分别为并行计算(95分)与数学建模(91分)。
\end{itemize}

\section{\faThumbsOUp\ 个人技能}
\begin{itemize}
	\item 会使用C、Python、Lua编程语言;
	\item 精通\LaTeX 排版系统;
	\item 熟练使用git进行代码管理;
	\item 托福成绩93分。
\end{itemize}

\section{\faLightbulbO\ 科研经历}
\datedsubsection{独立研究}{2020.01-2021.03}
Research on Litter Decomposition Based on the Olson Model 收录在5th International Conference on Economic Management and Green Development
会议期刊上;

\datedsubsection{本科科研项目}{2022.06-2023.05}
在何辉教授指导下,参与随机分支游走计数测度收敛速度估计的研究。

\datedsubsection{北京市创新项目}{2023.06-2024.05}
在陈昕昕教授指导下,参与分支布朗运动水平集大偏差问题的研究。

\section{\faBell\ 校内实践}
\datedsubsection{社团活动}{2021.10-2022.08}
\begin{itemize}
	\item OM学社成员,积极参与数学竞赛相关活动;
	\item 数科心理部成员,组织策划多项心理健康活动。
\end{itemize}

\datedsubsection{卓越训练营}{2021.10-2022.08}
成功参与并完成北京师范大学卓越训练营项目,锻炼了学术与领导力能力。

\datedsubsection{雪绒花使者}{2022}
完成校内志愿者培训,成为校内雪绒花使者,积极为同学们提供帮助。

\section{\faHeartO\ 获奖情况}
\datedline{\nth{1} 获得新生奖学金}{2019年}
\datedline{\nth{2} 获得京师二等奖学金}{2022年}
\datedline{\nth{3} 获得全国大学生数学竞赛三等奖}{2023年}
\datedline{\nth{4} 获得竞赛二等奖学金}{2023年}

\section{\faInfo\ 其他}
\begin{itemize}[parsep=0.5ex]
	\item \faGithub: \href{http://github.com/evawyy}{http://github.com/evawyy}
	\item 语言能力: 英语(流利),中文(母语)。
\end{itemize}

%% Reference
%\newpage
%\bibliographystyle{IEEETran}
%\bibliography{mycite}
\end{document}
